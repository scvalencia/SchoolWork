\documentclass[paper=a4, fontsize=11pt, spanish]{scrartcl}

\usepackage[spanish]{babel}
\usepackage[utf8]{inputenc}
\selectlanguage{spanish}
\usepackage{amsmath,amssymb,amsfonts}
\usepackage{color,linegoal}

\date{}

\definecolor{SolutionColor}{gray}{0.85}

\usepackage[T1]{fontenc} % Use 8-bit encoding that has 256 glyphs
\usepackage{fourier}
\usepackage{amsmath,amsfonts,amsthm} % Math packages

\usepackage{lipsum} % Used for inserting dummy 'Lorem ipsum' text into the template

\usepackage{sectsty} % Allows customizing section commands
\allsectionsfont{\centering \normalfont\scshape} % Make all sections centered, the default font and small caps

\usepackage{fancyhdr} % Custom headers and footers

\renewcommand{\headrulewidth}{0pt} % Remove header underlines
\renewcommand{\footrulewidth}{0pt} % Remove footer underlines
\setlength{\headheight}{13.6pt} % Customize the height of the header

\numberwithin{equation}{section} % Number equations within sections (i.e. 1.1, 1.2, 2.1, 2.2 instead of 1, 2, 3, 4)
\numberwithin{figure}{section} % Number figures within sections (i.e. 1.1, 1.2, 2.1, 2.2 instead of 1, 2, 3, 4)
\numberwithin{table}{section} % Number tables within sections (i.e. 1.1, 1.2, 2.1, 2.2 instead of 1, 2, 3, 4)

\setlength\parindent{0pt} % Removes all indentation from paragraphs - comment this line for an assignment with lots of text

%----------------------------------------------------------------------------------------
%	TITLE SECTION
%----------------------------------------------------------------------------------------

\newcommand{\horrule}[1]{\rule{\linewidth}{#1}} 

\title{	
\normalfont \normalsize 
\huge  Piezas escuchadas \\
\horrule{2pt}
}

\begin{document}
\maketitle

\section{Primer parcial}
\horrule{0.5pt}

\begin{itemize}

\item \textbf{Claude Debussy}, (1903), \textit{Estampes, Jardins sous la pluie}.\textsc{ Estampas - Jardines bajo la lluvia}. %%{Estampas - Jardines bajo la lluvia}
\item \textbf{Claude Debussy}, (1910), \textit{Preludes pour piano, La fille aux cheveux de lin}.\textsc{ Preludios para piano - La niña de los cabellos de lino}. %%{Preludios para piano - La nina de los cabellos de lino}
\item \textbf{Erik Satie}, (1913), \textit{Descriptions automatiques, Sur un vaisseau}. \textsc{ Descripciones automáticas - De un navío}. %%{Descripciones automaticas - De un navio}
\item \textbf{Erik Satie}, (1913), \textit{Descriptions automatiques, Sur une lanterne}.\textsc{ Descripciones automáticas - De una linterna}. %%{Descripciones automaticas - De una linterna}
\item \textbf{Erik Satie}, (1913), \textit{Descriptions automatiques, Sur un casque}.\textsc{ Descripciones automáticas - De un casco}. %%{Descripciones automaticas - De un casco}
\item \textbf{Erik Satie}, (1913), \textit{Croquis et Agaceries d'un Gros Bonhomme en Bois, Tyrolienne Turque}. \textsc{ Croquis y carantoñas de un gran hombre de madera - Tirolesa Turca}. %%{Croquis y carantonas de un gran hombre de madera - Tirolesa Turca}
\item \textbf{Erik Satie}, (1913), \textit{Croquis et Agaceries d'un Gros Bonhomme en Bois, Danse Maigre}. \textsc{ Croquis y carantoñas de un gran hombre de madera - Danza Flaca}. %%{Croquis y carantonas de un gran hombre de madera - Danza Flaca}
\item \textbf{Erik Satie}, (1913), \textit{Croquis et Agaceries d'un Gros Bonhomme en Bois, Espanana}. \textsc{ Croquis y carantoñas de un gran hombre de madera - Espanana}. %%{Croquis y carantonas de un gran hombre de madera - España}
\item \textbf{Erik Satie}, (1913), \textit{Embryons desseches, D'holothurie}.\textsc{ Embriones disecados - De holoturios}. %%{Embriones disecados - De holoturios}
\item \textbf{Erik Satie}, (1913), \textit{Embryons desseches, De podophthalma}.\textsc{ Embriones disecados - De edrioftalmos}. %%{Embriones disecados - De edrioftalmos}
\item \textbf{Erik Satie}, (1913), \textit{Embryons desseches, De podophthalma}.\textsc{ Embriones disecados - De podoftalmos}. %%{Embriones disecados - De podoftalmos}
\item \textbf{Erik Satie}, (1917), \textit{Sonatine bureaucratique, Allegro}.\textsc{ Sonatina burocrática - Allegro}. %%{Sonatina burocratica - Allegro}
\item \textbf{Erik Satie}, (1917), \textit{Sonatine bureaucratique, Andante}.\textsc{ Sonatina burocrática - Andante}. %%{Sonatina burocratica - Andante}
\item \textbf{Erik Satie}, (1917), \textit{Sonatine bureaucratique, Vivache}.\textsc{ Sonatina burocrática - Vivache}. %%{Sonatina burocratica - Vivache}
\item \textbf{Darius Milhaud}, (1937), \textit{Scaramouche, Vif}.\textsc{ Scaramouche - Vif}. %%{Scaramouche - Vif}
\item \textbf{Darius Milhaud}, (1937), \textit{Scaramouche, Modere}.\textsc{ Scaramouche - Moderato}. %%{Scaramouche - Moderato}
\item \textbf{Darius Milhaud}, (1937), \textit{Scaramouche, Brasilei}.\textsc{ Scaramouche - Brazileira}. %%{Scaramouche - Brazileira}

\end{itemize}

\section{Segundo parcial}
\horrule{0.5pt}

\begin{itemize}

\item \textbf{Arnold Schoenberg}, (1912), \textit{Pierrot Lunaire, Mondestrunken}. \textsc{ Pierrot Lunaire, Ebrio de Luna}. %%{Pierrot Lunaire - Ebrio de luna}
\item \textbf{Arnold Schoenberg}, (1912), \textit{Pierrot Lunaire, Columbine}. \textsc{ Pierrot Lunaire, Colombina}. %%{Pierrot Lunaire - Colombina}
\item \textbf{Arnold Schoenberg}, (1912), \textit{Pierrot Lunaire, Der Dandy}. \textsc{ Pierrot Lunaire, El Dandy}. %%{Pierrot Lunaire - El Dandy}
\item \textbf{Arnold Schoenberg}, (1912), \textit{Pierrot Lunaire, Eine blasse Waescherin}. \textsc{ Pierrot Lunaire, Una lavadera palida}. %%{Pierrot Lunaire - Una lavadera palida}
\item \textbf{Arnold Schoenberg}, (1912), \textit{Pierrot Lunaire, Valse de Chopin}. \textsc{ Pierrot Lunaire, Vals de Chopin}. %%{Pierrot Lunaire - Vals de Chopin}
\item \textbf{Arnold Schoenberg}, (1912), \textit{Pierrot Lunaire, Madonna}. \textsc{ Pierrot Lunaire, Madre dolorosa}. %%{Pierrot Lunaire - Madre dolorosa}
\item \textbf{Arnold Schoenberg}, (1912), \textit{Pierrot Lunaire, Der kranke Mond}. \textsc{ Pierrot Lunaire, La luna enferma}. %%{Pierrot Lunaire - La luna enferma}
\item \textbf{Alban Berg}, (1922), \textit{Wozzeck, Act 1 - Langsam Wozzeck Langsam!} \textsc{ Wozzeck, Acto 1, ¡Lento Wozzeck lento!} %%{Wozzeck - Acto 1 : ¡Lento Wozzeck lento!}
\item \textbf{Alban Berg}, (1922), \textit{Wozzeck, Act 1 - Du der Platz ist verflucht!} \textsc{ Wozzeck, Acto 1, ¡Tú la plaza está maldita!} %%{Wozzeck - Acto 1 : ¡Tu la plaza esta maldita!}
\item \textbf{Alban Berg}, (1922), \textit{Wozzeck, Act 1 - Tschin Bum Tschin Bum!} \textsc{ Wozzeck, Acto 1 : Tschin Bum Tschin Bum}. %%{Wozzeck - Acto 1 : Tschin Bum Tschin Bum}
\item \textbf{Alban Berg}, (1922), \textit{Wozzeck, Act 2 - Was die Steine glaenzen?} \textsc{ Wozzeck, Acto 2 : ¿Lo que las piedras brillan?} %%{Wozzeck - Acto 2 : ¿Lo que las piedras brillan?}
\item \textbf{Alban Berg}, (1922), \textit{Wozzeck, Act 2 - Was Hast Da}. \textsc{ Wozzeck, Acto 2 : ¿Qué tienes ahí?} %%{Wozzeck - Acto 2 : ¿Que tienes ahi? }
\item \textbf{Alban Berg}, (1922), \textit{Wozzeck, Act 2 - He Wozzeck!} \textsc{ Wozzeck, Acto 2 : ¡Hey, Wozzeck!} %%{Wozzeck - Acto 2 : ¡Hey Wozzeck!}
\item \textbf{Alban Berg}, (1922), \textit{Wozzeck, Act 3 - Dort links geht's in die Stadt}. \textsc{ Wozzeck, Acto 3 : A la izquierda está la ciudad}. %%{Wozzeck - Acto 3 :  A la izquierda esta  la ciudad}
\item \textbf{Alban Berg}, (1922), \textit{Wozzeck, Act 3 - Tanzt Al}. \textsc{ Wozzeck, Acto 3 : Bailen todos}. %%{Wozzeck - Acto 3 : Bailen todos}
\item \textbf{Alban Berg}, (1922), \textit{Wozzeck, Act 3 - Das messer? Wo ist das Messer?} \textsc{ Wozzeck, Acto 2 : El cuchillo, ¿dónde está el cuchillo?} %%{Wozzeck - Acto 3 : El cuchillo}
\item \textbf{Alban Berg}, (1935), \textit{Lulu, Act 1 - Scene 2 - Was tut denn Ihr Vater da?}\textsc{ Lulu, Acto 1, Escena 2 : Sonata}. %%{Lulu - Acto 1 : Escena 2 : Sonata}
\item \textbf{Alban Berg}, (1935), \textit{Lulu, Act 2 - Scene 1 - Wenn sich die Menschen}.\textsc{ Lulu - Acto 2 - Escena 1 - Cancion de Lulú}. %%{Lulu - Acto 2 : Escena 1 : Cancion de Lulu }
\item \textbf{Alban Berg}, (1935), \textit{Lulu, Act 3 - Scene 1 - Brilliant! Es geht brilliant}.\textsc{ Lulu - Acto 3 - Escena 1 - Conjunto 2}. %%{Lulu - Acto 3 : Escena 1 : Conjunto 2 }
\item \textbf{Alban Berg}, (1935), \textit{Lulu, Act 3 - Scene 2 - Wer Ist Das}.\textsc{ Lulu - Acto 3 - Escena 2 - Cuadro 4}. %%{Lulu - Acto 3 : Escena 2 : Cuadro 4 }
\item \textbf{Alban Berg}, (1935), \textit{Lulu, Act 3 - Scene 2 - Das ist der letzte Abend}.\textsc{ Lulu - Acto 3 - Escena 2 - Nocturno}. %%{Lulu - Acto 3 : Escena 2 : Nocturno }






\end{itemize}
\end{document}