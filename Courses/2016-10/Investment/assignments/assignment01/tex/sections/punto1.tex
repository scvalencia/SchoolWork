\documentclass[../main.tex]{subfiles}

\begin{document}
\begin{enumerate}[(a)]

\item \textbf{(8 puntos)} Para las siguientes situaciones indique qué preferiría. Justifique su respuesta de manera teórica.

\begin{enumerate}[(I)]

\item Prestar dinero a una tasa de interés simple del $X$ \%, o a una tasa de interé compuesto del $X$ \%.

\item Pedir dinero a una tasa de interés $X$ \% NA/SV, o a una tasa de $X$ \% NA/TV.

\item Pedir un préstamo en el que le cobran una tasa de interés de $X$ \% NA/MV, o a una tasa de $X$ \% NA/MA.
\item Prestar dinero a una tasa de interés $X$ \% NA/AV, o a una tasa de $X$ \% EA.

\end{enumerate}

\item \textbf{(3 puntos)} Fernando Macías es un desarrollador de software que recientemente ha alcanzado alto prestigio en el mercado por lo que dos empresas del sector le han ofrecido irse a trabajar con ellos. Actualmente Fernando gana \$4.459.000 pesos y la empresa A ofrece pagarle \$5.200.000 pesos en caso que decida trabajar con ellos. Análogamente, la empresa B ofrece pagarle \$4.200.000 pesos. ¿Cuál es el costo de oportunidad de Fernando Macías si acepta la oferta A? ¿Si acepta la oferta B?¿Si decide quedarse?

\item \textbf{(3 puntos)} Para completar el pago de la matrícula universitaria, a Carlos le hace falta \$1.000.000 pesos. Él sabe que dentro de 3 meses tendrá \$2.000.000 pesos. Sin embargo, para poder iniciar las clases  él debe efectuar el pago de la matrícula hoy. Carlos tiene dos opciones, la primera pedir prestado un millón de pesos en la Olla S.A. (prestamistas gota a gota) cerca a su casa quienes le cobran un interés compuesto de 5\% mensual. La segunda opción es financiar con la universidad quien le cobra un interés simple de 5 \% mensual. Para cualquiera de las dos opciones, Carlos deberá pagar al témino de los tres meses el capital más los intereses respectivos. ¿Cuál de las dos alternativas debería seleccionar Carlos de tal forma que pague lo menos posible por concepto de interés?

\item \textbf{(3 puntos)} Un amigo de la universidad le está ofreciendo una oportunidad de negocio. Él le propone una rentabilidad del 4,5 \% capitalización continua anual por el dinero que usted le entregue. ¿Cúal sería su rendimiento efectivo en téminos porcentuales sobre una inversión de \$1.000.000 pesos por un periodo de 3 años? Bajo las mismas condiciones, ¿Cúal sería su rendimiento efectivo en términos porcentuales sobre una inversión de \$67.760.000 pesos por un periodo de 3 años?, y ¿Cúal de las dos alternativas genera un mayor rendimiento efectivo en términos porcentuales?

\item \textbf{(2 puntos)} La familia Zeeman solicitó un préstamo por un valor de \$$X$ a la cooperativa Cooperando. La entidad prestadora le ofrecío a la familia una tasa de interés de $i$ \% SV, con un plazo de 4 semestres para amortizar el total de la deuda por medio de una cuota uniforme semestral, cada una por un monto \$Y.

\begin{enumerate}[(I)]

\item ¿Qué pasaría con la cuota uniforme semestral que debe pagar la familia Zeeman si la tasa de interés se redujera a la mitad de su valor? ¿Es mayor, menor o igual a la mitad de la cuota original?

\item ¿Qué pasaría con la cuota uniforme semestral que debe pagar Familia Zeeman (con respecto al pago semestral inicial de \$$Y$) si la tasa de interés cambia de $i$ \% SV a $i$ \% NA/SV? ¿El nuevo pago uniforme semestral sería mayor o menor al inicialmente contemplado?

\end{enumerate}

\item \textbf{(5 puntos)} Keyla Chiapana quiere comprar un televisor de 32”, cuyo precio hoy es \$1.375.987 pesos, por lo que va al amacén de cadena ‘Success S.A’ y allí tiene las siguientes formas de pago:

\begin{enumerate}[(1)]

\item Comprar el televisor de contado.

\item Comprar el televisor a 40 cuotas mensuales de \$35.000 pesos.

\item Comprar el televisor dando una cuota inicial de \$500.000 pesos y pagar el resto a 40 cuotas mensuales de \$22.000 pesos.

\item Comprar el televisor pidiendo el dinero prestado a un amigo, el cual quiere que se le regrese el dinero (que corresponde exactamente al valor del televisor el día de hoy) dentro de seis meses más una invitación a cenar (el día de la devolución del dinero) la cual tiene valor de \$102.599 pesos.

\end{enumerate}

Teniendo en cuenta que el costo de oportunidad de Keyla es 2\% EA, ¿cúal alternativa debería escoger?

\end{enumerate}

\end{document}
