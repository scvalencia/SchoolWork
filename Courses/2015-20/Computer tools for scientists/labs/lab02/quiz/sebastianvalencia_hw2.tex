\documentclass{article}
\title{Unix \textbf{WC(1)} manual: BSD General Commands Manual}
\author{Brian Kernighan}
\date{July 1971}

\begin{document}
\maketitle

\section{Name}
\textbf{wc} -- word, line, character, and byte count

\section{Synopsis}
\textbf{wc} [-clmw] [file ...]

\section{Description}
The \textbf{WC} utility displays the number of lines, words, and bytes contained
in each input file, or standard input (if no file is specified) to the
standard output.  A line is defined as a string of characters delimited
by a \textless newline\textgreater \ character.  Characters beyond the final  
\textless newline\textgreater \ character will not be included in the line count.\\

A word is defined as a string of characters delimited by white space
characters.  White space characters are the set of characters for which
the iswspace(3) function returns true.  If more than one input file is
specified, a line of cumulative counts for all the files is displayed on
a separate line after the output for the last file.\\

The following options are available:

\begin{description}
  	\item[-L] The number	of characters in the longest input line	is written to
	     the standard output.  When	more then one file argument is speci-
	     fied, the longest input line of all files is reported as the
	     value of the final	``total''.
  	\item[-c]	     The number	of bytes in each input file is written to the standard
	     output.  This will	cancel out any prior usage of the -m option.
  	\item[-l]	     The number	of lines in each input file is written to the standard
	     output.
	\item[-m]	     The number	of characters in each input file is written to the
	     standard output.  If the current locale does not support multi-
	     byte characters, this is equivalent to the	-c option.  This will
	     cancel out	any prior usage	of the -c option.
	\item[-w]	     The number	of words in each input file is written to the standard
	     output.
\end{description}

When an option is specified, wc only reports the information requested by
that option.  The order of	output always takes the	form of	line, word,
byte, and file name.  The default action is equivalent to specifying the
-c, -l and	-w options.\\

If	no files are specified,	the standard input is used and no file name is
displayed.	 The prompt will accept	input until receiving EOF, or  in
most environments.

\section{History}
A \textbf{wc} command appeared in Version 1	AT\&T UNIX.

\section{Author}
Written by Paul Rubin and David MacKenzie.

\section{Reporting Bugs}
Report \textbf{wc} bugs to bug-coreutils@gnu.org\\
GNU coreutils home page: \underline{http://www.gnu.org/software/coreutils}\\
General help using GNU software: \underline{http://www.gnu.org/gethelp}\\

\section{Copyright}
Copyright 2009 Free Software Foundation, Inc.   License  GPLv3+:  GNU\\
GPL version 3 or later \underline{http://gnu.org/licenses/gpl.html}.\\
This  is  free  software:  you  are free to change and redistribute it.\\
There is NO WARRANTY, to the extent permitted by law.\\

\section{See also}
The full documentation for wc is maintained as a  Texinfo  manual.   If
the  info  and  \textbf{wc}  programs  are  properly installed at your site, the
command

\end{document}