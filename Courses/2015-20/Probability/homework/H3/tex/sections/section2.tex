\documentclass[../main.tex]{subfiles}

\begin{document}
\begin{enumerate}[(a)]

\item Establezca la función de distribución de probabilidad conjunta de X y Y.

Primero se encuentra $F(X|Y) \sim Bin(p = 0.7; n = X, k = Y)$.

\begin{center}

\begin{tabular}{l*{3}{c}r}
$F(X|Y)$         & 1 & 2 & 3 & 4 \\
\hline
1              & 0.7 & 0 & 0 & 0  \\
2              & 0.42 & 0.49 & 0 & 0  \\
3              & 0.189 & 0.441 & 0.343 & 0  \\
4              & 0.0756 & 0.2646 & 0.4116 & 0.2401
\end{tabular}

\end{center}

\[ g_{X}(x) = \begin{cases} 
      \frac{\binom{8}{1} \times \binom{4}{3}}{\binom{12}{4}} = 0.065 & x = 1\\
      \frac{\binom{8}{2} \times \binom{4}{2}}{\binom{12}{4}} = 0.340 & x = 2\\
      \frac{\binom{8}{3} \times \binom{4}{1}}{\binom{12}{4}} = 0.450 & x = 3\\
      \frac{\binom{8}{4} \times \binom{4}{0}}{\binom{12}{4}} = 0.140 & x = 4
   \end{cases}
\]

$$f_{XY}(x, y) = f(Y | X) \times g_{X}(x)$$

\begin{center}

\begin{tabular}{l*{3}{c}r}
$f(y, x)$         & 1 & 2 & 3 & 4 \\
\hline
1              & 0.0455 & 0 & 0 & 0  \\
2              & 0.1428 & 0.1666 & 0 & 0  \\
3              & 0.08505 & 0.198 & 0.1543 & 0  \\
4              & 0.0105 & 0.037 & 0.057 & 0.0336
\end{tabular}

\end{center}

\item Calcule las funciones de probabilidad marginal para las variables
aleatorias X y Y.

\[ g_{X}(x) = \begin{cases} 
      0.065 & x = 1\\
      0.340 & x = 2\\
      0.450 & x = 3\\
      = 0.140 & x = 4
   \end{cases}
\]

De manera análoga: 

\[ h_{Y}(y) = \begin{cases} 
      0.284 & x = 1\\
      0.402 & x = 2\\
      0.211 & x = 3\\
      0.033 & x = 4
   \end{cases}
\]

\pagebreak

\item Calcule el valor esperado para cada una de las variables aleatorias e
interprete sus resultados en términos de la situación descrita.

$$\mathbb{E}[X] = \sum_{R[X]} x \times g_{X}(x) = 2.655$$

$$\mathbb{E}[Y] = \sum_{R[Y]} y \times h_{Y}(y) = 1.855$$

\item Determine si las variables aleatorias X y Y son independientes.

$$f(x, y) \stackrel{?}{=} g_{X}(x) \times h_{Y}(y) \Rightarrow $$
$$0.0455 \stackrel{?}{=} 0.065 \times 0.0284 \Rightarrow  0.0455 \neq 0.065 \times 0.0284 \ \text{Corporación 1}$$
$$0.1543 \stackrel{?}{=} 0.45 \times 0.2113 \Rightarrow  0.1543 \neq 0.45 \times 0.2113 \ \text{Corporación 2}$$

Las variables $X$ y $Y$ no son independientes.

\item Calcule el coeficiente de correlación entre el número de marcadores
usados que Juan selecciona de la caja de doce y el número de marcadores usados
que dejan de funcionar durante la clase complementaria. Interprete este resultado.

$$\rho_{XY} = \frac{Cov(X, Y)}{\sigma_{X} \times \sigma_{Y}}$$

Covarianza: 

$$Cov(X, Y) = \mathbb{E}(XY) -  \mathbb{E}(X) \mathbb{E}(Y)$$

$$\mathbb{E}(XY) = \sum_{R[Y]}\sum_{R[X]} xy \times f_{XY}(x, y) = 5.388$$

$$Cov(X, Y) = 5.388 -  (2.655)(1.855) = 0.46$$

Varianzas:

$$ \mathbb{V}ar(X)= \mathbb{E}(X^2) - [\mathbb{E}(X)]^2 = \sum_{R[X]}x^2 \times g_{X}(x) - 2.655^2 = 7.715 - 2.655^2 = 0.666$$

$$ \mathbb{V}ar(Y)= \mathbb{E}(Y^2) - [\mathbb{E}(Y)]^2 = \sum_{R[Y]}y^2 \times h_{Y}(y) - 1.855^2 = 4.33 - 2.655^2 = 0.888$$

Coeficiente de correlación:

$$\rho_{XY} = \frac{0.46}{\sqrt{0.666} \times \sqrt{0.888}} = 0.59$$

\item Si se sabe que Juan escogió 3 marcadores viejos entre los cuatro
escogidos para dictar la clase, ¿cuántos marcadores se esperaría que fallen durante
la clase complementaria?

$$f(Y | X = 3) = \sum_{R[Y]} y \times f(Y | X = 3) = 2.1$$

Se espera que fallen 2.1 marcadores.

\item Calcule la probabilidad de que el número de marcadores que fallan
durante la clase complementaria sea inferior al número de marcadores usados que
Juan toma de la caja de doce.
$$\mathbb{P}(Y < X) = \ \text{Sumatoria de los valores de } f_{XY}(x, y) \ \text{tales que } Y < X$$
$$ = f_{XY}(1, 2) + f_{XY}(2, 3) + f_{XY}(3, 4) + f_{XY}(1, 3) + f_{XY}(1, 4) + f_{XY}(2, 4) $$
$$ = 0.1428 + 0.08505 + 0.0105 + 0.198 + 0.037 + 0.057 = 0.53 $$

\item Si se sabe que el número de marcadores que fallaron era igual al número
de marcadores usados que Juan tomó de la caja de doce, determine la probabilidad
de que el número de marcadores usados que tomó juan de la caja sea igual a 2.

$$f_{XY}(Y = 2 | X = 2) = 0.49$$

\end{enumerate}
\end{document}
