\documentclass[../main.tex]{subfiles}

\begin{document}
\begin{enumerate}[(a)]

\item \textbf{(4 puntos)} Construya una prueba de hipótesis para determinar si las varianzas
poblacionales de los tiempos de producción de las dos líneas son diferentes. Plantee la
hipótesis nula y alterna, el estadístico de prueba, la región de rechazo y concluya en términos
del problema. Utilice un nivel de significancia del 5\%.

\item \textbf{(4 puntos)} Teniendo en cuenta el resultado anterior, plantee una prueba de hipótesis para comprobar si el jefe de producción tiene razón sobre la diferencia del tiempo entre las líneas de producción. Escriba las hipótesis nula y alterna, el estadístico de prueba y concluya en términos del problema. Utilice un nivel del significancia del 10\%, y utilice el criterio del p-valor para evaluar la prueba. Adicionalmente, el jefe de planta desea evaluar la línea de producción de botellas de vidrio únicamente, ya que considera que ésta línea presenta un tiempo de producción mayor a 22 segundos/botella, lo cual representaría una reducción considerable en la cantidad de botellas de vidrio producidas para suplir la demanda del próximo mes.

\item \textbf{(4 puntos)} Suponga que se conoce la varianza poblacional del tiempo que se demora un producto en salir de la línea de producción de botellas de vidrio ($\sigma ^2$ = $2.4^2$ ). Plantee una prueba de hipótesis para comprobar si el jefe de planta tiene razón sobre el tiempo de producción en la línea de botellas de vidrio. Plantee las hipótesis nula y alterna, el estadístico de prueba, la región de rechazo y concluya en términos del problema. Utilice un nivel de significancia del 10\%.

\item \textbf{(4 puntos)}  El jefe de planta tiene dudas sobre la varianza poblacional de la línea de producción de botellas de vidrio ya que considera que es mayor a la establecida en el punto
anterior. Plantee las hipótesis nula y alterna, el estadístico de prueba, la región de rechazo
y concluya en términos del problema. Utilice un nivel de significancia del 5\%.

\item \textbf{(4 puntos)} Asuma ahora que la información acerca de la varianza poblacional del tiempo que se demora un producto en salir de la línea de producción de botellas de vidrio es
descartada. Teniendo esto en cuenta, plantee nuevamente una prueba de hipótesis para
probar si el jefe de planta tiene razón sobre el tiempo de producción en dicha línea. Escriba
las hipótesis nula y alterna, el estadístico de prueba, la región de rechazo y concluya en
términos del problema. Utilice un nivel de significancia del 10\%.

\end{enumerate}
\end{document}
