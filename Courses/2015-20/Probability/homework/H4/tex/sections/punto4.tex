\documentclass[../main.tex]{subfiles}

\begin{document}
\begin{enumerate}[(a)]

\item \textbf{(4 puntos)} Calcule un intervalo de confianza del 95\% para la media del tiempo de procesamiento de cada uno de los tornos. Interprete cada uno de los intervalos de confianza.
Interprete este resultado.

Es necesario obtener los valores para $S_Y$ y $H$ de manera que sea posible construir el intervalo de confianza.

$$S_X = \sqrt{\frac{\sum(x_i - \bar{X})}{n - 1}} \,\ \bar{X} = \frac{1}{n} \times \sum x_i \Rightarrow$$

$$S_{XA} = 1.1 \wedge \bar{X}_A = 1.413 \wedge S_{XB} = 0.8 \wedge \bar{X}_B = 1.68$$
$$t_{(9.975)} = 2.086$$

El intervalo para el torno $A$ es:

$$\left( 1.413 - t_{(9.975)} \times \frac{S_{XA}}{\sqrt{21}} \leq 1.413 \leq 1.413 + t_{(9.975)} \times \frac{S_{XA}}{\sqrt{21}}\right)$$
$$(0.91 \leq 1.413 \leq 1.91)$$

El intervalo para el torno $B$ es:

$$\left( 1.68 - t_{(9.975)} \times \frac{S_{XB}}{\sqrt{21}} \leq 1.68 \leq 1.68 + t_{(9.975)} \times \frac{S_{XB}}{\sqrt{21}}\right)$$
$$(1.35 \leq 1.68 \leq 2.04)$$

\item \textbf{(4 puntos)} Calcule un intervalo de confianza del 90\% para la varianza del tiempo de procesamiento de cada uno de los tornos. Interprete este resultado.

Dado que la confianza es del 90\%, se tienen los siguientes parámetros:

$$\alpha = 0.1 \Rightarrow \mathcal{X}_{(0.95)} = 31.41 \wedge \mathcal{X}_{(0.05)} = 10.85$$

El intervalo para el torno $A$ es:

$$\left( \frac{(1.1)^2(20)}{(31.41)^2} \leq \sigma^2 \leq \frac{(1.1)^2(20)}{10.85} \right) \Rightarrow \left( 0.77 \leq \sigma ^2 \leq 2.23 \right)$$

El intervalo para el torno $B$ es:

$$\left( \frac{(0.8)^2(20)}{(31.41)^2} \leq \sigma^2 \leq \frac{(0.8)^2(20)}{10.85} \right) \Rightarrow \left( 0.407 \leq \sigma ^2 \leq 1.18 \right)$$

\item \textbf{(4 puntos)} Construya un intervalo de confianza del 90\% que le permita evaluar si las varianzas del tiempo de procesamiento de cada uno de los tornos son estadísticamente
diferentes. ¿Qué puede concluir?

Para evaluar la diferencia estadística de la diferencia de dos varianzas, se debe construir un intervalo de confianza para su cociente. $\alpha = 0.1 \Rightarrow \mathcal{F}_{((\alpha / 2) = 0.05)} = 0.47 \wedge \mathcal{F}_{((1 - \alpha / 2) = 0.95)} = 2.12 $.

$$\left[ 0.65 \leq \frac{\sigma_A ^2}{\sigma_B ^2} \leq 2.91 \right]$$

Dado que uno no está en el intervalo, con un 95\% de confianza, las varianzas son distintas en el intervalo de confianza.  

\item \textbf{(4 puntos)} Teniendo en cuenta la conclusión obtenida en el literal anterior, construya un intervalo de confianza del 95\% para la diferencia de las medias de procesamiento de ambos tornos ($\mu_X - \mu_Y$ ). ¿Qué puede concluir?

$t_{(1- \alpha / 2) = 0.975} = 2.02$

$$1.68 - 1.413 - \sqrt{\frac{2}{21} \times \left[ \frac{(0.8)^2(20)+(1.1)^2(20)}{21 + 21 - 2}\right]} \leq \mu_B - \mu_A$$
$$\mu_B - \mu_A \leq 1.68 - 1.413 + \sqrt{\frac{2}{21} \times \left[ \frac{(0.8)^2(20)+(1.1)^2(20)}{21 + 21 - 2}\right]} $$
$$(-0.33 \leq \mu_B - \mu_A \leq 0.866)$$

Con confianza del 95\%, no es posible afirmar si un tiempo es ayor al otro.
\end{enumerate}
\end{document}
