\documentclass[12pt]{article}
\usepackage{amsmath}
\usepackage{amsfonts}
\usepackage{amssymb}
\usepackage{mathabx}
\usepackage[margin=4cm]{geometry}
\usepackage{fancybox}
% Default margins are too wide all the way around.  I reset them here
\setlength{\topmargin}{-.5in}
\setlength{\textheight}{9in}
\setlength{\oddsidemargin}{.125in}
\setlength{\textwidth}{6.25in}
\newenvironment{wideverbatim}%
{\vskip\baselineskip\VerbatimEnvironment
\begin{Sbox}\begin{BVerbatim}}
{\end{BVerbatim}%
\end{Sbox}\noindent\centerline{\TheSbox}\vskip\baselineskip}
\begin{document}
\title{Tarea 3 se\~nales y sistemas}
\author{Sebasti\'an Valencia Calder\'on\\
Universidad de los Andes}
\renewcommand{\today}{Febrero 15, 2013}
\maketitle


\begin{enumerate}
\item  Encontrar la convoluci\'on circular discreta de los vectores $
\boldsymbol{\vec{x}}=[1, -1, 0, 0, 0, 0, 0, 0, 0, 0] $ y $
\boldsymbol{\vec{y}}=[10, 11, 12, 13, 14, 15, 16, 17, 18, 19] $.



La convoluci\'on circular de dos se\~nales discretas $x(t)$ y $y(t)$, se
define como la operaci\'on $ x(t) \oasterisk y(t)$ o de manera m\'as
espec\'ifica, la convoluci\'on circular de dos se\~nales $x(t)$ y $y(t)$, se
define como una se\~nal $g(t)$ definida de la siguiente manera: 
\

$$ g(t) = x(t) \oasterisk y(t) = \sum_{k=0}^{N-1} (x_k)(y_{t-k} \bmod N) $$

\

Donde $N$ es la longitud de los vectores a multiplicar, en \'este caso, las
se\~nales, poseen longitud $N=10$; la representaci\'on gr\'afica de las
se\~nales es bastante intuitiva, por lo tanto, se usar\'a un poco de
abstracci\'on para \'este ejercicio. La estrategia de soluci\'on en este caso,
depende de una propiedad importante de la concoluci\'on circular, la
conmutatividad (%nota al pie de p�gina
               Las propiedades m\'as significativas de la convoluci\'on
               circular se encuentran en:
               http://cnx.org/content/m12055/latest/), es decir, para todo par
               de se\~nales a convolucionar, se tiene:
$x(t) \oasterisk y(t) = y(t) \oasterisk x(t)$, por lo tanto, al ser pr\'acticos
se puede tomar como se\~nal variante la que contiene m\'as ceros, en este caso,
esa se\~nal corresponde a $
\boldsymbol{\vec{x}}=[1, -1, 0, 0, 0, 0, 0, 0, 0, 0] $, para aplicar la
convoluci\'on circular a este par de se\~nales $ \boldsymbol{\vec{x}} $ y
$\boldsymbol{\vec{y}}$, se pueden realizar ($N-1$) reflexiones circulares y el 
mismo n\'umero de ``shifts'' o corrimentos en el tiempo de la se\~nal no
est\'atica; lo descrito anteriormente, proporciona gran fuerza y vigor para
soluciones m\'as gr\'aficas o intuitivas, sin embargo, el enfoque que aqu\'i se
da es m\'as computacional y algor\'itmica. 
               %\footnote[number]{text}
\\
\\
La diguiente tabla, proporciona una regal de asignaci\'on para una se�al
evaluada sobre un m\'odulo $N$, es decir, la regla de asignaci\'on para una
funci\'on $h[t]_N$ para distintas elecciones de $t$. (%nota al pie de p�gina
               Una idea m\'as primitiva de \'este algoritmo, se encuentra en:
               http://cnx.org/content/m10786/latest/)

\begin{table}[ht]
% title of Table
\centering % used for centering table
\begin{tabular}{c c} % centered columns (4 columns)
\hline\hline %inserts double horizontal lines
$t$ & $h[t]_N$  \\ [0.5ex] % inserts table
%heading
\hline % inserts single horizontal line
$\vdots$ & $\vdots$ \\
$-m$ & $h[N-m]$  \\
$\vdots$ & $\vdots$ \\
0 & $h[0]$ \\
1 & $h[1]$ \\
2 & $h[2]$ \\
$\vdots$ & $\vdots$ \\
$N+m$ & $h[m]$  \\
$\vdots$ & $\vdots$ \\[1ex] % [1ex] adds vertical space
\hline %inserts single line
\end{tabular}
\label{table:nonlin} % is used to refer this table in the text
\caption{Valores de $h[t]$ de una se\~nal $\bmod N$}
\end{table}
\pagebreak

Como puede verse, la tabla 1.1, proporciona una asignaci\'on al valor presente
de la se\~nal seg\'un sean los valores de la misma en un intervalo $[0,N-1]$,
intervalo propicio para la evaluaci\'on de $ x(t) \oasterisk y(t)$.
\\
\\
Para \'este caso, se tiene que la convoluci\'on y su computaci\'on son:
$$ \boldsymbol{\vec{g(t)}} = \boldsymbol{\vec{x}} \oasterisk \boldsymbol{\vec{y}} =
\sum_{k=0}^{9} (x_k)(y_{t-k} \bmod 10) $$
$$ \boldsymbol{\vec{g(t)}} = 
(x_0)(y_{t-0} \bmod 10)+
(x_1)(y_{t-1} \bmod 10)+
(x_2)(y_{t-2} \bmod 10)+
(x_3)(y_{t-3} \bmod 10)+
 $$
 $$
(x_4)(y_{t-4} \bmod 10)+
(x_5)(y_{t-5} \bmod 10)+
(x_6)(y_{t-6} \bmod 10)+$$
$$(x_7)(y_{t-7} \bmod 10)+
(x_8)(y_{t-8} \bmod 10)+
(x_9)(y_{t-9} \bmod 10)$$
\

Si la tabla 1.1 se aplica para $N=10$, es f\'acil ver que para todo valor de 
$m$ en cierto conjunto, el valor de $h[t]$ ser\'a cero, \'este conjunto es
obviamente $ \mathbb{J}=[-8,-1] \cup [1,8]$, es decir, pata todo n\'umero en
\'este conjunto, el valor de su evaluaci\'on en $h[t]$ ser\'a cero. Lo anterior
se explica por la naturaleza c\'iclica de la tabla y de la se\~nal $x[t] =
h[t]$, pues si $h[t] \neq h[0]$ o $h[t] \neq h[1]$, entonces $ h[t] = 0 $, por
lo tanto, si se eval\'ua $ \boldsymbol{\vec{g(t)}} $ para $ t \in [0, N-1] $, se puede hacer
la siguiente simplificaci\'on.
$$g(0)=x(9)h(1)+x(0)h(0)=19-10=-9$$
$$g(1)=x(0)h(1)+x(1)h(0)=10-11=1$$
$$g(2)=x(1)h(1)+x(2)h(0)=11-12=1$$
$$g(3)=x(2)h(1)+x(3)h(0)=12-13=1$$
$$g(4)=x(3)h(1)+x(4)h(0)=13-14=1$$
$$g(5)=x(4)h(1)+x(5)h(0)=14-15=1$$
$$g(6)=x(5)h(1)+x(6)h(0)=15-16=1$$
$$g(7)=x(6)h(1)+x(7)h(0)=16-17=1$$
$$g(8)=x(7)h(1)+x(8)h(0)=17-18=1$$
$$g(9)=x(8)h(1)+x(9)h(0)=18-19=1$$

La simetr\'ia de la se\~nal resultante, es consecuencia de la naturalea
peri\'odica de la tabla 1.1 para $N=10$. Finalmente, la se\~nal
$\boldsymbol{\vec{g(t)}}$, est\'a dada por el siguiente vector
$$\boldsymbol{\vec{g(t)}}=[-9, 1, 1, 1, 1, 1, 1, 1, 1, 1]$$

Lo \'ultimo puede ser comprobado utilizando la siguiente rutina de MATLAB,
aplicando la funci\'on cconv(a,b,n). (%nota al pie de p�gina
               La documentaci\'on precisa de la
               funci\'on cconv de MATLAB, se encuentra
               en:http://www.mathworks.com/help/signal/ref/cconv.html)
\small{
\begin{verbatim}
>> a=[1 -1 0 0 0 0 0 0 0 0];
>> b=[10:19];
>> cconv(a,b,10)
ans =
     -9.0000    1.0000    1.0000    1.0000      
      1.0000    1.0000    1.0000    1.0000    
      1.0000    1.0000
\end{verbatim}
}
\item Calcular y graficar la DFT de la se\~nal 
$
\boldsymbol{\vec{x}}=[0, 1, 0] $
\end{enumerate}

La transformada discreta de Fourier, se define como un endomorfismo biyectivo
lineal con forma general $\alpha\colon \mathbb{C}^N \to \mathbb{C}^N$. Su
f\'ormula explicita, est\'a dada por la siguiente expresi\'on, para la cu\'al,
se tiene que $N$ es la longitud del vector de entrada y salida, por tratarse de
un endomorfismo.

Sea $\boldsymbol{\vec{x_k}} \in \mathbb{C}^N :
\alpha(\boldsymbol{\vec{x_k}})=\boldsymbol{\vec{X_k}} = \sum\limits_{n=0}^{N-1}
x_n e^{{\frac{-i2n\pi}{N}}k}, k \in [0,N-1]$, sea $\boldsymbol{\vec{x_k}}=[0,
1, 0]$, por lo tanto, el valor correspondiente de $N$ para \'este caso es 3,
luego, la forma de la transformada DFT, estar\'a dada por la siguiente serie:$$
\sum\limits_{n=0}^{N-1} x_n e^{{\frac{-i2n\pi}{3}}k} = \sum\limits_{n=0}^{2}
x_n e^{{\frac{-i2n\pi}{3}}k} = x_0 e^{{\frac{0\pi}{3}}k} + x_1
e^{{\frac{-i2\pi}{3}}k} +x_2 e^{{\frac{-i4\pi}{3}}k} $$ Reemplazando
directamente las posiciones de cada $x_n$ en la f\'ormula, se obtiene:
$ \boldsymbol{\vec{X_k}} = e^{{\frac{-i2\pi}{3}}k} $, por lo tanto,
$\boldsymbol{\vec{X_k}}$ es un vector cuya posici\'on $k$-\'esima corresponde a
$e^{{\frac{-i2\pi}{3}}k}$; como el valor de $k$ est\'a acotado, la computaci\'on
de su resultado no resulta tediosa, de tal manera, el vector transformado
ser\'a:
\pagebreak

$$k=0 \Rightarrow \boldsymbol{\vec{X_k}} = \boldsymbol{\vec{X_0}} =
e^{{\frac{-i2\pi}{3}}(0) = 1}$$

$$k=1 \Rightarrow \boldsymbol{\vec{X_k}} = \boldsymbol{\vec{X_1}} =
e^{{\frac{-i2\pi}{3}}(1)} = e^{\frac{-i2\pi}{3}} $$

$$k=2 \Rightarrow \boldsymbol{\vec{X_k}} = \boldsymbol{\vec{X_2}} =
e^{{\frac{-i2\pi}{3}}(2)} = e^{\frac{-i4\pi}{3}}$$

La diversidad, versatilidad y grandeza de las matem\'aticas permiten la
posibilidad de m\'ultiples representaciones gr\'aficas para \'estos puntos, sin embargo, se
selecciona por simplicidad una representaci\'on sencilla en el plano complejo
seg\'un la forma polar y por solidez una representaci\'on amplitud y fase de
la respuesta o DFT del vector. (Las representaciones se encuentran en la
\'ultima p\'agina). Para obtener la representaci\'on amplitud fase, se utiliz\'o
el siguiente c\'odigo:

\small{
\begin{wideverbatim}
>> N=3;
>> a=[0 1 0];
>> b=zeros(N);
>> for k=1:N
     for n=1:N
         w=exp((-2*pi*i*(k-1)*(n-1))/N);
         x(n)=w;
     end
     c(k,:)=x;
  end
>> r=[c]*[a']
r =   1.0000       -0.5000 - 0.8660i -0.5000 + 0.8660i
>> subplot(2,1,1)
>> stem(abs(r));
>> subplot(2,1,2)
>> stem(angle(r));
\end{wideverbatim}
}

En la representaci\'on polar, los puntos resaltados representan cada componente
de la DFT, en la segunda representaci\'on, se tiene la amplitud y fase de cada
componente de la DFT del vector.
%\footnote{}
%begin{verbatim}
%\end{wideverbatim}
%\pagebreak
%\begin{wideverbatim}
%\end{wideverbatim}%

\end{document}