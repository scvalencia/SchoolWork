\documentclass[12pt]{article}
\usepackage{amsmath}
\usepackage{amsfonts}
\usepackage{amssymb}
\usepackage{mathabx}
\usepackage[margin=4cm]{geometry}
\usepackage{fancybox}
% Default margins are too wide all the way around.  I reset them here
\setlength{\topmargin}{-.5in}
\setlength{\textheight}{9in}
\setlength{\oddsidemargin}{.125in}
\setlength{\textwidth}{6.25in}
\newenvironment{wideverbatim}%
{\vskip\baselineskip\VerbatimEnvironment
\begin{Sbox}\begin{BVerbatim}}
{\end{BVerbatim}%
\end{Sbox}\noindent\centerline{\TheSbox}\vskip\baselineskip}
\begin{document}
\title{Tarea 2. C\'alculo de predicados}
\author{Sebasti\'an Valencia Calder\'on\\
Universidad de los Andes}
\renewcommand{\today}{Marzo, 2013}
\maketitle

\bf{Ejercicio 1.}\rm Exprese cada expresi\'on en lenguaje natural usando la
sint\'axis de la l\'ogica de primer orden. Suponga que el dominio es el conjunto
de estudiantes de la clase. Considere los siguientes predicados
\\
\\
\bf{G(x):}\rm x tiene un gato. \\
\bf{P(x):}\rm x tiene un perro.\\
\bf{C(x):}\rm x tiene una chinchilla.\\ \\
Para la siguiente soluci\'on, supongase que se representa el universo como
\bf{U}\rm=$\Omega\setminus(x,y,z)$, donde $\Omega$ es el alfabeto del lenguaje
natural, entonces, se tiene que el universo es un conjunto igual a $\Omega$
diferencia sim\'etrica $x$, $y$ y $z$; estas \'ultimas variables ser\'an las
variables aleatorias o excluidas de \bf{U}\rm, por lo que se tiene $x, y, z=a
\vee b \vee \ldots \vee w$. Es claro que no necesariamente se establece una diferencia estricta en las variables
excluidas.
\begin{enumerate}
  \item S\'olamente un estudiante en la clase tiene un perro y una chinchilla. 
$$H(x)=  C(x) \wedge P(x) $$
$$H(y)=H(x)[x:=y]$$
$$(\exists x,y \mid H(x) \wedge H(y) \Rightarrow x=y )$$
 \item Hay estudiantes de la clase que no tiene ni un perro ni un gato.
$$(\exists x \mid \neg P(x) \wedge \neg G(x) )$$
 \item Hay estudiantes que no tiene perro ni gato pero si tienen chinchilla.
$$(\exists x \mid \neg P(x) \wedge \neg G(x) \wedge C(x))$$
\item Ning\'un estudiante tiene las tres mascotas.
$$\neg (\exists x \mid P(x) \wedge G(x) \wedge C(x))$$
\item Para cada uno de los tres tipos de animales hay un estudiante que lo tiene
como mascota.
$$((\exists x \mid C(x)) \wedge (\exists y \mid P(y)) \wedge (\exists z \mid
G(z)))$$
\end{enumerate}

\bf{Ejercicio 2.}\rm Demuestre el siguiente teorema

$$(\forall x \mid R: Q \vee P)\wedge R[x:=a] \wedge \neg Q[x:=a] \Rightarrow
P[x:=a]$$ 
Demostraci\'on:
$$(\forall x \mid R: Q \vee P)$$
$$\equiv  \langle \mbox{\bf{Se asume true y el antecedente de la
implicaci\'on}\rm} \rangle$$ 
$$\neg(\neg (\forall x \mid R: Q \vee P))$$
$$\equiv  \langle \mbox{\bf{Doble negaci\'on y asociatividad}\rm} \rangle$$
$$\neg(\exists x \mid R: \neg (Q \vee P))$$
$$\equiv  \langle \mbox{\bf{De de Morgan generalizada 3}\rm} \rangle$$
$$\neg(\exists x \mid R: \neg Q \wedge \neg P))$$
$$\equiv  \langle \mbox{\bf{De de Morgan sobre $\vee$}\rm} \rangle$$
$$\neg(\exists x \mid R \wedge \neg Q :\neg P))$$
$$\equiv  \langle \mbox{\bf{Trading de $\exists$ 1}\rm} \rangle$$
$$(\forall x \mid R \wedge \neg Q : P)$$
$$\equiv  \langle \mbox{\bf{De de Morgan generalizada 2}\rm} \rangle$$
$$(\forall x \mid R(x) \wedge \neg Q(x) : P(x))$$
$$\equiv  \langle \mbox{\bf{Equivalencia sem\'antica}\rm} \rangle$$
$$R(a) \wedge \neg Q(a) : P(a)$$
$$\equiv  \langle \mbox{\bf{a $\in$ x}\rm} \rangle$$
$$R(a) \wedge \neg Q(a) \Rightarrow P(a)$$
$$\equiv  \langle \mbox{\bf{Equivalencia sem\'antica}\rm} \rangle$$
$$(\forall x \mid R: Q \vee P) \wedge R(a) \wedge \neg Q(a) \Rightarrow P(a) $$
$$\equiv  \langle \mbox{\bf{Hip\'otesis}\rm} \rangle$$
$$\Rightarrow P(a) $$
$$\equiv  \langle \mbox{\bf{Modus ponens antecedente supuesto y paso
anterior}\rm} \rangle$$
$$\Rightarrow P(a) $$
\\
\\
\\

\bf{Ejercicio 3.}\rm Considere el siguiente enunciado:\\
Todos los republicanos que pertenecen al Tea party est\'an en contra de un
programa de amnist\'ia para ilegales. Marco Rubio es republicano pero est\'a a
favor de un programa de amnist\'ia para ilegales. Por lo tanto Marco Rubio no
pertenece al Tea Party.

Suponga que el dominio son todos los pol\'iticos norte americanos.
\begin{enumerate}
  \item Traduzca a la l\'ogica de predicados
  \item Es posible comcluir la \'ultima frase de las anteriores, muestre o
  refute.
\end{enumerate}

Soluci\'on:

Sea \bf{U}\rm el universo del problema, y la codificaci\'on de predicados se
muestra a continuaci\'on: \bf{R(x)}\rm para x es republicano, \bf{TP(x)}\rm para
x es del tea party, \bf{AAI(x)}\rm para x apoya el programa de amnist\'ia para
ilegales.\\
Todos los republicanos que pertenecen al Tea party est\'an en contra de un
programa de amnist\'ia para ilegales, se puede codificar correctamente como 
$$(\forall x \in U \mid R(x) \wedge TP(x): \neg AAI(x))$$
Por el primer Trading de la cuantificaci\'on universal, se tiene:
$$(\forall x \in U \mid R(x): TP(x) \Rightarrow \neg AAI(x))$$
Por contrapositiva, se tiene:
$$(\forall x \in U \mid R(x): AAI(x)  \Rightarrow \neg TP(x) )$$
Nuevamente por el primer Trading de la cuantificaci\'on universal, se tiene:
$$(\forall x \in U \mid R(x) \wedge AAI(x): \neg TP(x))$$
Sea $mr$ Marco Rubio, tal que $mr \in U$, luego es cierto por el predicado
anterior que:
$$R(mr) \wedge AAI(mr): \neg TP(mr)$$
La traducci\'on al lenguaje natural da:
Si Marco Rubio es republicano y est\'a a favor de un
programa de amnist\'ia para ilegales, entonces, Marco Rubio no pertenece al tea
party.
  
 





\end{document}