\documentclass{article}
\usepackage{amssymb}
\usepackage{amsmath}
\usepackage{dsfont}
\usepackage{tikz}
\usetikzlibrary{shapes,arrows,shapes.multipart}
\usepackage{alltt}
\usepackage[spanish,activeacute]{babel}
\usepackage{algorithm}% http://ctan.org/pkg/algorithms
\usepackage{algpseudocode}% http://ctan.org/pkg/algorithmicx
\addtolength{\textwidth}{1.0in}
\addtolength{\textheight}{1.00in}
\addtolength{\evensidemargin}{-0.75in}
\addtolength{\oddsidemargin}{-0.75in}
\addtolength{\topmargin}{-.50in}
\newtheorem{theorem}{Theorem}
\newenvironment{proof}{\noindent{\bf Proof:}}{$\hfill \Box$ \vspace{10pt}}  
\begin{document}
\title{Matem\'atica estructural y l\'ogica \\ Inducci\'on y estructuras bien
ordenadas.}
\author{Sebasti\'an Valencia Calder\'on}
\date{Mayo 2013}
\maketitle

\textbf{Ejercicio 1.} Demuestre que para todo n\'umero natural $n$, 5 divide a
$F_{5n}$.
$$ ( \forall n: \mathbb{N} \mid : 5 \mid F_{5n} ) $$
{\raggedright
La demostraci\'on se realizar\'a usando el principio de inducci\'on 
matem\'atica, referido desde ahora como PIM. Para esto, se tomar\'a como
caso base el n\'umero cero, luego se suponr\'a el predicado para un natural
cualquiera, y posteriormente, se estructura la demostraci\'on con PMI.
$\text{Caso base:} \quad P(0) :  5 \mid F_{0} = 5 \mid 0 \equiv \text{TRUE}$, 
$\text{Hip\'otesis inductiva:} \quad ( \forall k: \mathbb{N} \mid : 5 \mid
F_{5k} )$
}
$\textbf{Lema 1:} \quad a \mid b \iff \frac{b}{a} \in \mathbb{Z}$ \\
$\textbf{L1 proof:} \quad a \mid b \iff b = k_{\mathbb{Z}}a \Rightarrow k \in
\mathbb{Z}$
\\
$\textbf{Lema 2:} \quad a \mid b \Rightarrow a \mid kb$ \\
$\textbf{L2 proof:} \quad a \mid b \Rightarrow \frac{b}{a} \in \mathbb{Z}, k \in
\mathbb{Z} \Rightarrow k\frac{b}{a} \in \mathbb{Z} \Rightarrow a \mid kb \vdash
\textbf{Lema 1 y clausura de $\langle \mathbb{Z},* \rangle$ } $
\\
$\textbf{Lema 3:} \quad a \mid b \wedge a \mid c \Rightarrow a \mid b+c$ \\
$\textbf{L3 proof:} \quad a \mid b \Rightarrow b = ak_{\mathbb{Z}};a \mid c
\Rightarrow c = am_{\mathbb{Z}}::b+c=a(k_{\mathbb{Z}}+m_{\mathbb{Z}})
\Rightarrow b+c = al_{\mathbb{Z}} \\ \vdash
\textbf{Definici\'on de divisibilidad y clausura de $\langle \mathbb{Z},+
\rangle$ } $
\\
\\
$ F_{5(k+1)} = F_{5(k+1)-1} + F_{5(k+1)-2} \vdash \text{Definici\'on
                Fibonacci} $\\
$  = F_{5k+4} + F_{5k+3} \vdash \text{Aritm\'etica} $ \\
$  = (F_{5k+3} + F_{5k+2}) + (F_{5k+2} + F_{5k+1})\vdash
    \text{Definici\'on de Fibonacci} $ \\
$  = F_{5k+3} + 2F_{5k+2} + F_{5k+1} \vdash \text{Aritm\'etica}  $\\
$  = (F_{5k+2} + F_{5k+1}) + 2(F_{5k+1} + F_{5k}) +F_{5k+1}\vdash 
    \text{Aritm\'etica y definici\'on de Fibonacci} $ \\
$  = (F_{5k+1} +F_{5k}+ F_{5k+1}) + 2F_{5k+1} + 2F_{5k}
     +F_{5k+1}\vdash \text{Aritm\'etica y definici\'on de Fibonacci} $ \\
$  = 5F_{5k+1} + 3F_{5k} \vdash\text{Aritm\'etica} $  \\
$ \Rightarrow ((5 \mid 5F_{5k+1}) \wedge (5 \mid 3F_{5k})) \vdash\text{Lema 1
sobre $5F_{5k+1}$, lema 2 e hip\'otesis inductiva sobre $\mid 3F_{5k}$ } $ \\
$ \Rightarrow 5 \mid (5F_{5k+1} + 3F_{5k}) \vdash \text{Lema 3 con
enunciado anterior} $\\
$ \Rightarrow 5 \mid F_{5(k+1)} \vdash \text{Sustituci\'on de expresi\'on
equivalente} $\\
$ \Rightarrow  \vdash \text{Por PMI, $P(n)$ es TRUE} $\\

\textbf{Ejercicio 2.} Demuestre el siguiente predicado.
$$P(n) : \equiv ( \forall n: \mathbb{N} \mid n \geq
1:F_{n+1}F_{n-1}-F_{n}^2=(-1)^n ) $$ 
{\raggedright
$\text{Caso base:} \quad P(1) : \equiv F_{2}F_{0}-F_{1}^2=(-1)^1  \equiv
-(1)^2=(-1)^1 \equiv \text{TRUE}$
} \\
{\raggedright
$\text{Hip\'otesis inductiva:} \quad (P(k) : \equiv  \forall k: \mathbb{N} \mid
k \geq 1:F_{k+1}F_{k-1}-F_{k}^2=(-1)^k $ 
} \\
{\raggedright
$\text{Posible consecuencia:} \quad P(k+1) : \stackrel{?}{\equiv}
F_{k+2}F_{k}-F_{k+1}^2=(-1)^{k+1} $ } \\
\\
$ F_{k+2}F_{k}-F_{k+1}^2 = (F_{k+1}+F_{k})F_{k}-F_{k+1}^2 \vdash
\text{Definici\'on Fibonacci} $\\
$  = F_{k}F_{k+1}+F_{k}^2-F_{k+1}^2 \vdash \text{Distributividad} $ \\
$  = F_{k}F_{k+1}+F_{k}^2-(F_{k}+F_{k-1})^2 \vdash \text{Def Fibonacci} $ \\
$  = F_{k}F_{k+1}+F_{k}^2-F_{k}^2-2F_{k-1}F_{k}-F_{k-1}^2 \vdash
     \text{Expansi\'on binomial} $ \\
$  = F_{k}F_{k+1}-2F_{k-1}F_{k}-F_{k-1}^2 \vdash
     \text{Inverso aditivo sobre $\langle \mathbb{N},+
\rangle$} $ \\
$  = F_{k}(F_{k}+F_{k-1})-2F_{k-1}F_{k}-F_{k-1}^2 \vdash
     \text{Def Fibonacci} $ \\
$  = F_{k}^2+F_{k}F_{k-1}-2F_{k-1}F_{k}-F_{k-1}^2 \vdash
     \text{Distributiva} $ \\
$  = F_{k}^2-F_{k-1}F_{k}-F_{k-1}^2 \vdash
     \text{\'Algebra} $ \\
$  = F_{k}^2-F_{k-1}(F_{k}-F_{k-1}) \vdash
     \text{\'Algebra, factorizaci\'on} $ \\
$  = F_{k}^2-F_{k-1}F_{k+1} \vdash
     \text{Def Fibonacci} $ \\
$  = (-1)(F_{k-1}F_{k+1}-F_{k}^2) \vdash
     \text{Inverso aditivo sobre $\langle \mathbb{N},+
\rangle$} $ \\
$  = (-1)(-1)^k \vdash
     \text{Hip\'otesis inductiva} $ \\
$  = (-1)^{k+1} \vdash
     \text{Propiedades potenciaci\'on} $ \\
$  \Rightarrow P(k+1),\text{Por PMI, $P(n)$ es cierto para todo $n$} $ \\

\textbf{Ejercicio 3.} Demuestre la siguiente expresi\'on.
$$ 	P(n) :\equiv \sum_{i=0}^{n}  (\frac{-1}{2}) ^ i = \frac{2^{n+1}+(-1)^n}{3
\times 2^n} $$
{\raggedright
$$\text{Caso base:} \quad P(0) :\equiv \sum_{i=0}^{0}  (\frac{-1}{2})^i = 1 =
\frac{2^{1}+1}{3 \times 1} = 1 \equiv \text{TRUE} $$
} \\
{\raggedright
$$\text{Hip\'otesis inductiva:} \quad P(k) :\equiv \sum_{i=0}^{k} 
(\frac{-1}{2}) ^ i = \frac{2^{k+1}+(-1)^k}{3 \times 2^k} $$
} \\
\begin{flalign}
        \sum_{i=0}^{k+1}  (\frac{-1}{2}) ^ i = (\frac{-1}{2}) ^ 0 +
        (\frac{-1}{2}) ^ 1 +\ldots+(\frac{-1}{2}) ^ {k-1}+(\frac{-1}{2}) ^
        {k}+(\frac{-1}{2}) ^ {k+1}\\
        (\sum_{i=0}^{k} (\frac{-1}{2}) ^ i ) + (\frac{-1}{2}) ^ {k+1} \\
         \frac{2^{k+1}+(-1)^k}{3 \times 2^k} + (\frac{-1}{2}) ^ {k+1} \vdash
        \text{Hip\'otesis inductiva}\\
        \frac{2^{k+1}+(-1)^k}{3 \times 2^k} + \frac{(-1)^{k+1}}{2^{k+1}}
        \vdash \text{\'Algebra}\\
        \frac{2^{k+1}(2^{k+1}+(-1)^{k})+(3 \times 2^k)(-1)^{k+1}}{3 \times 2^k
        \times 2^{k+1}} \vdash \text{\'Algebra}\\
        \frac{2^{k}  \lbrack  2(2^{k+1}+(-1)^{k})+3(-1)^{k+1}
        \rbrack }{3 \times 2^k \times 2^{k+1}} \vdash \text{Factorizaci\'on}\\
        \frac{  2(2^{k+1}+(-1)^{k})+3(-1)^{k+1}
         }{3 \times 2^{k+1}} \vdash \text{Simplificaci\'on}\\
         \frac{  2^{k+2}+(-1)^{k}\lbrack 2+3(-1)\rbrack
         }{3 \times 2^{k+1}} \vdash \text{Factorizaaci\'on}\\
         \frac{  2^{k+2}+(-1)^{k}(-1)}{3 \times 2^{k+1}} \vdash
         \text{Aritm\'etica}\\
         \frac{  2^{k+2}+(-1)^{k+1}}{3 \times 2^{k+1}} \vdash
         \text{Aritm\'etica}\\
         \Rightarrow \text{P(n) por PMI}
\end{flalign}
\pagebreak
\\
\textbf{Ejercicio 4.} Demuestre el siguiente teorema.
$$ 	dlast(S \rhd x) = S $$
{\raggedright
$\text{Caso base:} \quad P(\epsilon) :\equiv (dlast(\epsilon \rhd x)) =
(dlast(x \triangleleft \epsilon)) = \epsilon \vdash \text{Axioma de anexar
con cadena vacia} \equiv \text{TRUE} $ }
{\raggedright
$\text{Hip\'otesis inductiva:} \quad P(S) :\equiv (\forall S,x \mid dlast(S
\rhd x) = S)$}

\begin{flalign}
        dlast((y \lhd S) \rhd x) = dlast(y \lhd (S \rhd x)) \vdash
        \text{Asociatividad de listas}\\
        dlast(y \lhd (S \rhd x)) = y \lhd dlast(S \rhd x)  \vdash
        \text{Definici\'on $dlast(x \lhd S)$} \\  
        y \lhd dlast(S \rhd x) = y \lhd S \vdash
        \text{Hip\'otesis inductiva} \\ 
        \Rightarrow  dlast((y \lhd S) \rhd x) = y \lhd S \Rightarrow P(y \lhd S)
        \vdash
        \text{PMI}
\end{flalign}
\textbf{Ejercicio 5.} Demuestre el siguiente teorema.
$$ 	asc(S \lhd x) \equiv asc(S) \wedge (x \leq min_{S}(S)) $$
{\raggedright
$\text{Caso base:} \quad P(\epsilon) :\equiv asc(S \lhd \epsilon) =
asc(\epsilon) \wedge (x \leq min(\epsilon)) = TRUE \wedge (x \leq \infty) \equiv
( TRUE \wedge TRUE ) \equiv TRUE \vdash \text{Definici\'on de $asc$ y $\infty$,
propiedades $\wedge$} $
} \\
\\
{\raggedright
$\text{Hip\'otesis inductiva:} \quad P(S) :\equiv (\forall S,x \mid asc(S \lhd x) \equiv asc(S) \wedge (x \leq min_{S}(S)))$}
\\
Por hip\'otesis inductiva, y simplificaci\'on sobre P($n$), $x \leq min_{S}(S)
$. Por definici\'on, $min_{S}(y \lhd S) = min(y,min_{S}(S))$. Como S es una
secuencia ascendente por P(n), $min_{S}(S) = car(S)$, y $min_{S}(y \lhd S) =
min(y,car(S))$, para que $asc(y \lhd S)$, sea cierto, $y \leq car(S)$, es decir
$y = min_{S}(y \lhd S)$.

\begin{flalign}
        asc(x \lhd (y \lhd S)) = x \leq y \wedge asc(y \lhd S) \vdash
        \text{Definici\'on de $asc$}\\
        = asc(y \lhd S) \wedge x \leq y \vdash
        \text{Simetr\'ia de $\wedge$}\\
        = asc(y \lhd S) \wedge x \leq min_{S}(y \lhd S) \vdash
        \text{Hip\'otesis inductiva y deducci\'on de arriba} \\
        \Rightarrow P(y \lhd S) \vdash \text{Por PMI, P(n) es cierto}      
\end{flalign}

\Sigma=\left[
\begin{array}{ccc}
   \sigma_{11} & \cdots & \sigma_{1n} \\
   \vdots & \ddots & \vdots \\
   \sigma_{n1} & \cdots & \sigma_{nn}
\end{array}
\right]






% Define block styles
\tikzset{
decision/.style = {diamond, draw, fill=blue!20, 
  text width=4.5em, text badly centered, node distance=3cm, inner sep=0pt},
block/.style = {rectangle, draw, fill=blue!20, 
  text width=5em, text centered, rounded corners, minimum height=4em},
line/.style = {draw, -latex'},
cloud/.style = {draw, ellipse,fill=red!20, node distance=3cm,
  minimum height=2em},
subroutine/.style = {draw,rectangle split, rectangle split horizontal,
  rectangle split parts=3,minimum height=1cm,
  rectangle split part fill={red!50, green!50, blue!20, yellow!50}},
connector/.style = {draw,circle,node distance=3cm,fill=yellow!20},
data/.style = {draw, trapezium,node distance=3cm,fill=olive!20}
}

\begin{tikzpicture}[node distance = 2cm, auto]
    % Place nodes
    \node [block] (init) {initialize model};
    \node [data, left of=init] (expert) {expert};
    \node [connector, right of=init] (system) {system};
    \node [block, below of=init] (identify) {identify candidate models};
    \node [block, below of=identify] (evaluate) {evaluate candidate models};
    \node [block, left of=evaluate, node distance=3cm] (update) {update     model};
    \node [decision, below of=evaluate] (decide) {is best candidate     better?};
    \node [block, below of=decide, node distance=3cm] (test) {test};
    \node [subroutine, below of=test, node distance=3cm] (sub) {part1\nodepart{two}part2\nodepart{three}part3};
    % Draw edges
    \path [line] (init) -- (identify);
    \path [line] (identify) -- (evaluate);
    \path [line] (evaluate) -- (decide);
    \path [line] (decide) -| node [near start] {yes} (update);
    \path [line] (update) |- (identify);
    \path [line] (decide) -- node {no}(test);
    \path [line,dashed] (test) -- (sub);
    \path [line,dashed] (expert) -- (init);
    \path [line,dashed] (system) -- (init);
    \path [line,dashed] (system) |- (evaluate);
\end{tikzpicture}


 







\end{document}
